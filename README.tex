\documentclass[12pt,a4paper,twoside]{report}
\usepackage{amsmath}
\usepackage{bm}
\usepackage[colorlinks=true,urlcolor=blue]{hyperref}
\usepackage[table]{colortbl}
\usepackage{graphicx}
\usepackage{fullpage}
\title{Snp HeRitability Estimation Kit (SHREK) Manual}
\date{\today}
\author{Choi Shing Wan, Sam\\
\texttt{choishingwan@gmail.com}}

\begin{document}

\maketitle
\tableofcontents
\chapter{Background}
\chapter{Installation}
\section{Dependencies}
The programme is dependent on the ISO C++ 2011 standard and Eigen library.
\section{Tutorial}
\chapter{Input formats}
\section{Test statistic file}
\section{Linkage file}
The linkage file should be in the binary ped format. 
\chapter{Examples}
\chapter{Assumptions and limitations}
\section{Assumptions}
\subsection{Calculation of variance}
There are multiple assumptions when we calculate the variance. 
First, we assume a big sample size. Only when the sample size is big, can we approximate the distribution of the effects to follow the $\chi^2$ distribution. 
Therefore, when the sample size is small, our variance estimate can be biased.

Secondly, for simplicity of implementation, when calculating the variance covariance matrix of the effects, we uses the \textbf{maximum sample size} as the $n$.
So, for example, if 9 out of 10 Snps' association were conducted on 100 samples, yet the remaining 1 Snp were conducted on 10,000 samples, the programme will still use $n= 10000$ in the calculation of the variance covariance of \textbf{all} the Snp pairs.
\section{Limitations}
\subsection{Direction of effect}
If one doesn't provide the direction of effect, the programme will use the direction from the test-statistic. 
However, if only p-values are provided, all effect will be assumed to be positive. 
This will lead to a positive bias in the variance explained.
Therefore it is advised to always provide the direction of effect.
\subsection{Perfect LD}
In the case where two or more Snps are in perfect LD, we cannot determine the true underlying heritability explained of each Snps.
Therefore, in SHREK, Snps in perfect LD will be grouped together and formed one Snp where its effect is the mean of all Snps in the group.
Decomposition will then be performed on this Snp and the resulting heritability will be \textbf{equally distributed} among the original group of Snps.

So for example:
Assuming there are 5 Snps, $Snp_1, Snp_2, Snp_3, Snp_4 and Snp_5$ where $Snp_1$, $Snp_2$ and $Snp_3$ are all in perfect LD each with effect $f_1$,$f_2$ and $f_3$ respectively.
Then we will first get $Snp_\mu$ with effect as $f_\mu = \frac{f_1+f_2+f_3}{3}$. 
As $Snp_1$, $Snp_2$ and $Snp_3$ has the same LD structure with all other Snps (as they are in perfect LD), we then use the LD structure of $Snp_1$ as a representative of the LD of $Snp_\mu$ with all other Snps. 
This will then gives us
\[ \left( \begin{array}{ccc}
R_{11} & R_{14} & R_{15}\\
R_{14} & R_{44} & R_{45}\\
R_{15} & R_{45} & R_{55}
\end{array} \right)
%
\left( \begin{array}{c}
h_\mu \\
h_d \\
h_e
\end{array} \right)
=
\left( \begin{array}{c}
f_\mu \\
f_d \\
f_e
\end{array} \right)
\]
Finally, we will get
\begin{align*}
h_1 &= \frac{h_\mu}{3} \\
h_2 &= \frac{h_\mu}{3} \\
h_3 &= \frac{h_\mu}{3}
\end{align*}

It is important to note that for the variance, it is \textbf{NOT} equally divided among the Snps. 
Instead, it is something like

\begin{align*}
var(h_1) &= var(h_\mu) \\
var(h_2) &= var(h_\mu) \\
var(h_3) &= var(h_\mu)
\end{align*}

\end{document}